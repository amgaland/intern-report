%----------------------------------------------------------------------------------------
%   Доорх хэсгийг өөрчлөх шаардлагагүй
%----------------------------------------------------------------------------------------
%!TEX TS-program = xelatex
%!TEX encoding = UTF-8 Unicode
\documentclass[12pt,A4]{report}

\usepackage{fontspec,xltxtra,xunicode}
\setmainfont[Ligatures=TeX]{Times New Roman}
\setsansfont{Arial}

% \usepackage[utf8x]{inputenc}
% \usepackage[mongolian]{babel}
%\usepackage{natbib}
\usepackage{geometry}
%\usepackage{fancyheadings} fancyheadings is obsolete: replaced by fancyhdr. JL
\usepackage{fancyhdr}
\usepackage{float}
\usepackage{afterpage}
\usepackage{graphicx}
\usepackage{amsmath,amssymb,amsbsy}
\usepackage{dcolumn,array}
\usepackage{tocloft}
\usepackage{dics}
\usepackage{nomencl}
\usepackage{upgreek}
\newcommand{\argmin}{\arg\!\min}
\usepackage{mathtools}
\usepackage[hidelinks]{hyperref}
\usepackage{bookmark}

\usepackage{algorithm}
\usepackage{algpseudocode}

\usepackage{listings}
\DeclarePairedDelimiter\abs{\lvert}{\rvert}%
\makeatletter
\usepackage{caption}
\captionsetup[table]{belowskip=0.5pt}
\usepackage{subfiles}

\usepackage{listings}
\renewcommand{\lstlistingname}{Код}
\renewcommand{\lstlistlistingname}{\lstlistingname ын жагсаалт}

\usepackage{color}
\definecolor{codegreen}{rgb}{0,0.6,0}
\definecolor{codegray}{rgb}{0.5,0.5,0.5}
\definecolor{codepurple}{rgb}{0.58,0,0.82}
\definecolor{backcolour}{rgb}{0.99,0.99,0.99}
\definecolor{darkgray}{rgb}{0.99,0.103,0.105}
\definecolor{purple}{rgb}{0.128,0,0.128}
 
\lstdefinestyle{mystyle}{
    basicstyle=\ttfamily\small,
    backgroundcolor=\color{backcolour},   
    commentstyle=\color{codegreen},
    keywordstyle=\color{magenta},
    numberstyle=\tiny\color{codegray},
    stringstyle=\color{codepurple},
    %basicstyle=\footnotesize,
    breakatwhitespace=false,         
    breaklines=true,                 
    captionpos=b,                    
    keepspaces=false,                 
    numbers=left,                    
    numbersep=10pt,                  
    showspaces=false,                
    showstringspaces=true,
    showtabs=false,                  
    tabsize=2
}

% Define TypeScript language for listings
\lstdefinelanguage{TypeScript}{
    keywords={as, abstract, any, argument, boolean, break, case, class, const, constructor, continue, debugger, default, delete, do, else, enum, export, extends, false, finally, for, from, function, get, if, implements, import, in, instanceof, interface, let, module, namespace, new, null, number, object, package, private, protected, public, readonly, return, static, string, super, switch, symbol, throw, true, try, type, undefined, var, void, while},
    keywordstyle=\color{blue}\bfseries,
    ndkeywords={boolean, throws, try, void, with, yield, this, import},
    ndkeywordstyle=\color{darkgray}\bfseries,
    identifierstyle=\color{black},
    sensitive=false,
    comment=[l]{//},
    morecomment=[s]{/*}{*/},
    commentstyle=\color{purple}\ttfamily,
    stringstyle=\color{red}\ttfamily,
    morestring=[b]',
    morestring=[b]"
}

% Define Go language for listings
\lstdefinelanguage{Go}{
    keywords={break, case, chan, const, continue, default, defer, else, fallthrough, for, func, go, goto, if, import, interface, map, package, range, return, select, struct, switch, type, var},
    keywordstyle=\color{blue}\bfseries,
    ndkeywords={interface, select, go, map},
    ndkeywordstyle=\color{darkgray}\bfseries,
    identifierstyle=\color{black},
    sensitive=true,
    comment=[l]{//},
    morecomment=[s]{/*}{*/},
    commentstyle=\color{purple}\ttfamily,
    stringstyle=\color{red}\ttfamily,
    morestring=[b]',
    morestring=[b]"
}

\lstset{
    extendedchars=true,
    basicstyle=\footnotesize\ttfamily,
    showstringspaces=false,
    showspaces=false,
    numbers=left,
    numberstyle=\footnotesize,
    numbersep=9pt,
    tabsize=2,
    breaklines=true,
    showtabs=false,
    captionpos=b
}

\lstset{style=mystyle, label=DescriptiveLabel} 

\let\oldabs\abs
\def\abs{\@ifstar{\oldabs}{\oldabs*}}
\makenomenclature
\begin{document}

%----------------------------------------------------------------------------------------
%   Өөрийн мэдээллээ оруулах хэсэг
%----------------------------------------------------------------------------------------

% Дипломийн ажлын сэдэв
\title{Веб хөгжүүлэлт}
% Дипломын ажлын англи нэр
\titleEng{Web developing}
% Өөрийн овог нэрийг бүтнээр нь бичнэ
\author{Баянжаргалын Энх-Амгалан}
% Өөрийн овгийн эхний үсэг нэрээ бичнэ
\authorShort{Б.Энх-Амгалан}
% Удирдагчийн зэрэг цол овгийн эхний үсэг нэр
\supervisor{Др. Ч.Алтангэрэл}
% Хамтарсан удирдагчийн зэрэг цол овгийн эхний үсэг нэр
\cosupervisor{Н.Од-Эрдэнэ}

% СиСи дугаар 
\sisiId{21B1NUM0344}
% Их сургуулийн нэр
\university{МОНГОЛ УЛСЫН ИХ СУРГУУЛЬ}
% Бүрэлдэхүүн сургуулийн нэр
\faculty{МЭДЭЭЛЛИЙН ТЕХНОЛОГИЙН ЭЛЕКТРОНИКИЙН СУРГУУЛЬ}
% Тэнхимийн нэр
\department{МЭДЭЭЛЭЛ, КОМПЬЮТЕРИЙН УХААНЫ ТЭНХИМ}
% Зэргийн нэр
\degreeName{Үйлдвэрлэлийн дадлагын тайлан}
% Суралцаж буй хөтөлбөрийн нэр
\programeName{Мэдээллийн технологи (D061303)}
% Хэвлэгдсэн газар
\cityName{Улаанбаатар}
% Хэвлэгдсэн огноо
\gradyear{2025 оны 01сар}

%----------------------------------------------------------------------------------------
%   Доорх хэсгийг өөрчлөх шаардлагагүй
%----------------------------------------------------------------------------------------
\include{important/main-pre}

\begin{abstract}
   Миний бие Б.Энх-Амгалан нь үйлдвэрлэлийн дадлагын хугацаанд Next.js, Golang гэсэн технологиуд дээр голчлон ажилласан ба уг технологиуд ямар шалтгаанаар үүссэн, цаана нь технологийн ямар дэвшил, хөгжүүлэлтийн арга барил ашигладаг, компаниуд хэрхэн үүн дээр хөгжүүлэлт хийж эцсийн бүтээгдэхүүнийг гаргадаг талаар судлахын тулд front-end хөгжүүлэлт дээрээ Next.js фрэймворк болон back-end хөгжүүлэлт дээрээ Golang ашигладаг компани болох "ОНДО" ХХК-г сонгон авч мэргэжлийн дадлагаа гүйцэтгэлээ.
  
  \quad \quad \textbf{Зорилго} Next.js болон Golang технологийн талаар судалж, компанийн хөгжүүлэлтийн арга барилтай танилцах
  
  \quad \quad \textbf{Зорилт} Удирдагчийн зааварчилгааны дагуу алхам алхмаар судалгаа хийж өгсөн шаардлагын хүрээнд хэрэгжүүлэлт хийх
\end{abstract}


\addcontentsline{toc}{part}{БҮЛГҮҮД}

\chapter{БАЙГУУЛЛАГЫН ТАНИЛЦУУЛГА}
\subfile{chapters/introduction}

\chapter{АЖЛЫН ТӨЛӨВЛӨГӨӨ}
\subfile{important/plan}

\chapter{СИСТЕМИЙН ШААРДЛАГА}
\subfile{chapters/requirements}

\chapter{АШИГЛАХ ТЕХНОЛОГИ}
\subfile{chapters/technologies}

\chapter{ХЭРЭГЖҮҮЛЭЛТ}
\subfile{chapters/implementation}

\chapter{ДҮГНЭЛТ}
\subfile{chapters/conclusion.tex}

%----------------------------------------------------------------------------------------
%   Дипломын номзүй, хавсралтын хэсэг эндээс эхэлнэ
%----------------------------------------------------------------------------------------

\singlespace
\addcontentsline{toc}{part}{НОМ ЗҮЙ}
\begin{thebibliography}{99}
	% Ашигласан материалыг эндээс оруулна
	\bibitem{Golang Docs}
	Golang Docs
	\\\url{https://go.dev/doc/}

	\bibitem{GORM Docs}
	GORM Docs
	\\\url{https://gorm.io/docs/index.html}

	\bibitem{NextJS Docs}
	NextJS Docs
	\\\url{https://nextjs.org/docs}

	\bibitem{Go дээр тест бичих заавар}
	Go дээр тест бичих заавар
	\\\url{https://go.dev/doc/tutorial/add-a-test}
\end{thebibliography}



%----------------------------------------------------------------------------------------
%   Хавсралтууд эндээс эхэлнэ
%----------------------------------------------------------------------------------------
\appendix
\addcontentsline{toc}{part}{ХАВСРАЛТ}

% Хавсралтын нэр. Хавсралт гэдэг үг агуулахгүй
\chapter{Удирдагчийн үнэлгээ}
\includegraphics[width=14cm]{images/review.jpg}
% Хавсралтын нэр. Хавсралт гэдэг үг агуулахгүй
\chapter{Кодын хэрэгжүүлэлт}

\section{GORM}

\subsection{Golang дээр Gorm aшигласан байдал}
\begin{lstlisting}[language=Go, frame=single]
	package services

	import (
		"branch-admin-service/config"
		"branch-admin-service/log"
		"branch-admin-service/models"
		"branch-admin-service/utils"
		"errors"
		"net/http"
	
		"github.com/gin-gonic/gin"
		"gorm.io/gorm"
	)
	
	func GetAllUsers(r *gin.Context) ([]models.User, error) {
		var users []models.User
		if err := config.DB.Find(&users).Error; err != nil {
			log.Error("Failed to fetch users", map[string]interface{}{"error": err.Error()}, r)
			return nil, err
		}
		return users, nil
	}
	
	func CreateUser(user models.User, r *http.Request) (models.User, error) {
		existingUser := config.DB.Where("login_id = ?", user.LoginID).First(&models.User{})
		if existingUser.RowsAffected > 0 {
			return models.User{}, errors.New("user already exists")
		}
	
		hashedPassword, err := utils.HashPassword(user.Password)
		if err != nil {
			return models.User{}, err
		}
		user.Password = hashedPassword
	
		if err := config.DB.Create(&user).Error; err != nil {
			return models.User{}, err
		}
		return user, nil
	}
	
	func UpdateUser(id string, user models.User, r *http.Request) (models.User, error) {
		if err := config.DB.Model(&user).Where("id = ?", id).Updates(user).Error; err != nil {
			return models.User{}, err
		}
		return user, nil
	}
	
	func DeleteUser(id string, r *http.Request) error {
		if err := config.DB.Where("user_id = ?", id).Delete(&models.UserRole{}).Error; err != nil {
			return err
		}
		if err := config.DB.Where("id = ?", id).Delete(&models.User{}).Error; err != nil {
			return err
		}
		return nil
	}
	
	func GetUserByID(id string, r *gin.Context) (*models.User, error) {
		log.Info("Fetching user by ID from the database", map[string]interface{}{"id": id}, r)
	
		var user models.User
		err := config.DB.Where("id = ?", id).First(&user).Error
		if err != nil {
			if errors.Is(err, gorm.ErrRecordNotFound) {
				log.Warn("User not found", map[string]interface{}{"id": id}, r)
				return nil, nil
			}
			log.Error("Database query error", map[string]interface{}{"error": err.Error()}, r)
			return nil, err
		}
	
		log.Info("User fetched successfully", map[string]interface{}{"id": id}, r)
		return &user, nil
	}
	
	func CheckLoginIDExists(loginID string, r *gin.Context) (bool, error) {
		var user models.User
		result := config.DB.Where("login_id = ?", loginID).First(&user)
	
		if result.Error != nil {
			if errors.Is(result.Error, gorm.ErrRecordNotFound) {
				return false, nil
			}
			log.Error("Failed to check login_id", map[string]interface{}{"error": result.Error.Error()}, r)
			return false, result.Error
		}
	
		return true, nil
	}
	
\end{lstlisting}

\section{Тест}
\subsection{Ашигласан GORM дээрээ тест бичсэн байдал}

\begin{lstlisting}[language=Go, frame=single]
	package service_test

	import (
		"log"
		"net/http"
		"net/http/httptest"
		"testing"
		"time"
	
		"branch-admin-service/config"
		"branch-admin-service/models"
		services "branch-admin-service/services/admin"
		test "branch-admin-service/test/utils"
	
		"github.com/stretchr/testify/assert"
		"gorm.io/gorm"
	)
	
	func setupTestDB() *gorm.DB {
		test.ConnectTestDatabase()
		config.DB.Exec("TRUNCATE TABLE users RESTART IDENTITY CASCADE") 
		return config.DB
	}
	
	
	func dropExistingUser(db *gorm.DB, loginID string) {
		var existingUser models.User
		if err := db.Where("login_id = ?", loginID).First(&existingUser).Error; err == nil {
			if err := db.Delete(&existingUser).Error; err != nil {
				log.Fatal("Error deleting existing user:", err)
			}
		}
	}
	
	func createSampleUser(loginID string) models.User {
		dummyDate := time.Date(2021, time.January, 1, 0, 0, 0, 0, time.UTC)
		emailPersonal := "personal@mail.com"
		phoneNumberWork := "9876543210"
		admin := "admin"
	
		return models.User{
			FirstName:           "John",
			LastName:            "Doe",
			LoginID:             loginID,
			EmailWork:           "test@mail.com",
			EmailPersonal:       &emailPersonal,
			PhoneNumberWork:     &phoneNumberWork,
			PhoneNumberPersonal: &phoneNumberWork,
			IsActive:            true,
			ActiveStartDate:     dummyDate,
			ActiveEndDate:       &dummyDate,
			Password:            "password123",
			CreatedAt:           dummyDate,
			UpdatedAt:           dummyDate,
			CreatedBy:           &admin,
			UpdatedBy:           &admin,
		}
	}
	
	func TestCreateUser(t *testing.T) {
		db := setupTestDB()
		defer test.CloseTestDatabase()
	
		loginID := "johndoe"
		dropExistingUser(db, loginID)
	
		r := httptest.NewRequest(http.MethodPost, "/", nil)
		user := createSampleUser(loginID)
	
		createdUser, err := services.CreateUser(user, r)
	
		assert.NoError(t, err, "Error creating user")
		assert.NotZero(t, createdUser.ID, "User ID should not be zero")
		assert.NotEqual(t, "password123", createdUser.Password, "Password should be hashed")
	
		var fetchedUser models.User
		err = db.First(&fetchedUser, "id = ?", createdUser.ID).Error
		assert.NoError(t, err, "Error fetching user")
	
		assert.Equal(t, "John", fetchedUser.FirstName, "First name mismatch")
		assert.Equal(t, "Doe", fetchedUser.LastName, "Last name mismatch")
		assert.NotEqual(t, "password123", fetchedUser.Password, "Password should be hashed")
	}
	
	func TestGetAllUsers(t *testing.T) {
		db := setupTestDB()
		defer test.CloseTestDatabase()
	
		db.Exec("TRUNCATE TABLE users RESTART IDENTITY CASCADE")
	
		users := []models.User{
			{FirstName: "John", LastName: "Doe"},
			{FirstName: "Jane", LastName: "Smith"},
		}
		for _, user := range users {
			db.Create(&user)
		}
	
		r := httptest.NewRequest(http.MethodGet, "/users", nil)
		allUsers, err := services.GetAllUsers(r)
	
		assert.NoError(t, err, "Error fetching all users")
		assert.Len(t, allUsers, len(users), "Unexpected number of users fetched")
	
		for i, fetchedUser := range allUsers {
			assert.True(t, i < len(users), "Index %d out of range for expected users", i)
			assert.Equal(t, users[i].FirstName, fetchedUser.FirstName, 
				"First name mismatch. Expected: %s, Got: %s", users[i].FirstName, fetchedUser.FirstName)
			assert.Equal(t, users[i].LastName, fetchedUser.LastName, 
				"Last name mismatch. Expected: %s, Got: %s", users[i].LastName, fetchedUser.LastName)
		}
	}
	
	
	
	func TestUpdateUser(t *testing.T) {
		db := setupTestDB()
		defer test.CloseTestDatabase()
	
		// Create sample user
		originalUser := createSampleUser("johndoe")
		db.Create(&originalUser)
	
		// Prepare updated user data
		updatedUserData := models.User{
			FirstName: "Jane",
			LastName:  "Smith",
			Password:  "newpassword123",
		}
	
		r := httptest.NewRequest(http.MethodPut, "/user", nil)
		updatedUser, err := services.UpdateUser(originalUser.ID, updatedUserData, r)
	
		assert.NoError(t, err, "Error updating user")
		assert.Equal(t, updatedUserData.FirstName, updatedUser.FirstName, "First name mismatch")
		assert.Equal(t, updatedUserData.LastName, updatedUser.LastName, "Last name mismatch")
		assert.NotEqual(t, "newpassword123", updatedUser.Password, "Password should be hashed")
	
		var fetchedUser models.User
		err = db.First(&fetchedUser, "id = ?", originalUser.ID).Error
		assert.NoError(t, err, "Error fetching updated user")
	
		assert.Equal(t, updatedUserData.FirstName, fetchedUser.FirstName, "First name mismatch in database")
		assert.Equal(t, updatedUserData.LastName, fetchedUser.LastName, "Last name mismatch in database")
		assert.NotEqual(t, "newpassword123", fetchedUser.Password, "Password should be hashed in database")
	}
	
	func TestDeleteUser(t *testing.T) {
		db := setupTestDB()
		defer test.CloseTestDatabase()
	
		// Create sample user
		user := createSampleUser("johndoe")
		db.Create(&user)
	
		r := httptest.NewRequest(http.MethodDelete, "/user", nil)
		err := services.DeleteUser(user.ID, r)
	
		assert.NoError(t, err, "Error deleting user")
	
		var fetchedUser models.User
		err = db.Where("id = ?", user.ID).First(&fetchedUser).Error
		assert.Error(t, err, "Expected error for fetching deleted user")
		assert.Equal(t, gorm.ErrRecordNotFound, err, "Expected record not found error for deleted user")
	}
	
	
	func TestGetUserByID(t *testing.T) {
		db := setupTestDB()
		defer test.CloseTestDatabase()
	
		db.Exec("TRUNCATE TABLE users RESTART IDENTITY CASCADE")
	
		sampleUser := models.User{FirstName: "John", LastName: "Doe", ID: "123"}
		db.Create(&sampleUser)
	
		t.Run("Valid ID", func(t *testing.T) {
			r := httptest.NewRequest(http.MethodGet, "/user/123", nil)
			user, err := services.GetUserByID("123", r)
	
			assert.NoError(t, err, "Error fetching user with valid ID")
			assert.NotNil(t, user, "User should not be nil for valid ID")
			assert.Equal(t, sampleUser.ID, user.ID, "User ID mismatch")
			assert.Equal(t, sampleUser.FirstName, user.FirstName, "User first name mismatch")
		})
	
		t.Run("Invalid ID", func(t *testing.T) {
			r := httptest.NewRequest(http.MethodGet, "/user/999", nil)
			user, err := services.GetUserByID("999", r)
	
			// Handle the error cleanly
			assert.NoError(t, err, "Error should be nil for non-existent user ID")
			assert.Nil(t, user, "Fetched user should be nil for non-existent ID")
		})
	}
\end{lstlisting}

\section{Front-end}

\subsection{/admin/user/edit/page.tsx}
\begin{lstlisting}[language=Typescript, frame=single]
	"use client";
	import { useEffect, useState } from "react";
	import { useRouter, useSearchParams } from "next/navigation";
	import { Input } from "@/components/ui/input";
	import { Button } from "@/components/ui/button";
	import { Label } from "@/components/ui/label";
	import { toast } from "@/hooks/use-toast";
	import { Card, CardHeader, CardTitle, CardContent } from "@/components/ui/card";
	import { req } from "@/app/api";
	import { useSession } from "next-auth/react";
	import { DatePickerWithRange } from "@/components/date-range-picker";
	import UserRole from "./components/user-role";
	
	interface UserFormData {
		first_name: string;
		last_name: string;
		login_id: string;
		email_work: string;
		email_personal: string;
		phone_number_work: string;
		phone_number_personal: string;
		is_active: boolean;
		active_start_date: string;
		active_end_date: string;
		password?: string;
		created_by: string;
		updated_by: string;
	}
	
	export default function UserFormPage() {
		const searchParams = useSearchParams();
		const id = searchParams.get("id");
		const router = useRouter();
		const { data: session } = useSession();
	
		const initialFormData: UserFormData = {
			first_name: "",
			last_name: "",
			login_id: "",
			email_work: "",
			email_personal: "",
			phone_number_work: "",
			phone_number_personal: "",
			is_active: false,
			active_start_date: "",
			active_end_date: "",
			password: "",
			created_by: session?.user?.id || "",
			updated_by: session?.user?.id || "",
		};
	
		const [formData, setFormData] = useState<UserFormData>(initialFormData);
		const [loading, setLoading] = useState(false);
		const [errors, setErrors] = useState<Record<string, string>>({});
	
		useEffect(() => {
			const fetchUserData = async () => {
				if (!id || !session?.user?.token) return;
	
				try {
					setLoading(true);
					const response = await req.GET(
						`/admin/users?id=${id}`,
						session.user.token
					);
					setFormData({
						...response,
						updated_by: session.user.id || "",
					});
				} catch (error) {
					toast({
						title: "Алдаа",
						description: "Хэрэглэгчийн мэдээлэл авахад алдаа гарлаа",
						variant: "destructive",
					});
				} finally {
					setLoading(false);
				}
			};
	
			fetchUserData();
		}, [id, session?.user?.id, session?.user?.token]);
	
		const handleChange = (e: React.ChangeEvent<HTMLInputElement>) => {
			const { name, value, type, checked } = e.target;
			setFormData((prev) => ({
				...prev,
				[name]: type === "checkbox" ? checked : value,
			}));
		};
	
		const handleSubmit = async (e: React.FormEvent) => {
			e.preventDefault();
	
			try {
				setLoading(true);
				const response = await req.PUT(
					`/admin/users/${id || ""}`,
					session?.user?.token || "",
					{
						...formData,
						active_start_date: formData.active_start_date || null,
						active_end_date: formData.active_end_date || null,
					}
				);
	
				if (response) {
					toast({
						title: "Амжилттай",
						description: "Мэдээлэл шинэчлэгдлээ",
					});
				}
			} catch (error) {
				console.error("Error submitting form:", error);
				toast({
					title: "Алдаа",
					description: "Гэнэтийн алдаа гарлаа",
					variant: "destructive",
				});
			} finally {
				setLoading(false);
			}
		};
	
		return (
			<div className="py-8 px-4">
				<Card>
					<CardHeader>
						<CardTitle className="text-2xl">
							{id ? "Хэрэглэгч засах" : "Шинэ хэрэглэгч нэмэх"}
						</CardTitle>
					</CardHeader>
					<CardContent>
						<form onSubmit={handleSubmit} className="space-y-4">
							<div className="space-y-6">
								<h3 className="text-lg font-medium">Хувийн мэдээлэл</h3>
								<div className="grid grid-cols-1 md:grid-cols-2 gap-4">
									<div>
										<Label htmlFor="first_name">Нэр</Label>
										<Input
											id="first_name"
											name="first_name"
											type="text"
											value={formData.first_name}
											onChange={handleChange}
											className={errors.first_name ? "border-red-500" : ""}
										/>
										{errors.first_name && (
											<p className="text-red-500 text-sm mt-1">
												{errors.first_name}
											</p>
										)}
									</div>
									<div>
										<Label htmlFor="last_name">Овог</Label>
										<Input
											id="last_name"
											name="last_name"
											type="text"
											value={formData.last_name}
											onChange={handleChange}
											className={errors.last_name ? "border-red-500" : ""}
										/>
										{errors.last_name && (
											<p className="text-red-500 text-sm mt-1">
												{errors.last_name}
											</p>
										)}
									</div>
									<div>
										<Label htmlFor="login_id">Нэвтрэх ID</Label>
										<Input
											id="login_id"
											name="login_id"
											type="text"
											value={formData.login_id}
											onChange={handleChange}
											className={errors.login_id ? "border-red-500" : ""}
										/>
										{errors.login_id && (
											<p className="text-red-500 text-sm mt-1">
												{errors.login_id}
											</p>
										)}
									</div>
									{!id && (
										<div>
											<Label htmlFor="password">Нууц үг</Label>
											<Input
												id="password"
												name="password"
												type="password"
												value={formData.password}
												onChange={handleChange}
												className={errors.password ? "border-red-500" : ""}
											/>
											{errors.password && (
												<p className="text-red-500 text-sm mt-1">
													{errors.password}
												</p>
											)}
										</div>
									)}
								</div>
							</div>
							<div className="space-y-2">
								<h3 className="text-lg font-medium">Холбоо барих</h3>
								<div className="grid grid-cols-1 md:grid-cols-2 gap-4">
									<div>
										<Label htmlFor="email_work">Ажлын имэйл</Label>
										<Input
											id="email_work"
											name="email_work"
											type="email"
											value={formData.email_work}
											onChange={handleChange}
											className={errors.email_work ? "border-red-500" : ""}
										/>
										{errors.email_work && (
											<p className="text-red-500 text-sm mt-1">
												{errors.email_work}
											</p>
										)}
									</div>
									<div>
										<Label htmlFor="email_personal">Хувийн имэйл</Label>
										<Input
											id="email_personal"
											name="email_personal"
											type="email"
											value={formData.email_personal}
											onChange={handleChange}
											className={errors.email_personal ? "border-red-500" : ""}
										/>
										{errors.email_personal && (
											<p className="text-red-500 text-sm mt-1">
												{errors.email_personal}
											</p>
										)}
									</div>
									<div>
										<Label htmlFor="phone_number_work">Ажлын дугаар</Label>
										<Input
											id="phone_number_work"
											name="phone_number_work"
											type="tel"
											value={formData.phone_number_work}
											onChange={handleChange}
										/>
									</div>
									<div>
										<Label htmlFor="phone_number_personal">Хувийн дугаар</Label>
										<Input
											id="phone_number_personal"
											name="phone_number_personal"
											type="tel"
											value={formData.phone_number_personal}
											onChange={handleChange}
										/>
									</div>
								</div>
							</div>
							<div className="space-y-2">
								<h3 className="text-lg font-medium">Төлөв</h3>
								<div className="grid grid-cols-1 md:grid-cols-2 gap-4 items-center">
									<div className="flex items-center space-x-2">
										<input
											id="is_active"
											name="is_active"
											type="checkbox"
											checked={formData.is_active}
											onChange={handleChange}
											className="h-4 w-4 border-gray-300 rounded"
										/>
										<Label htmlFor="is_active">Идэвхтэй эсэх</Label>
									</div>
									<div className="space-x-2">
										<Label>Ажиллах хугацаа:</Label>
										<DatePickerWithRange
											value={{
												from: formData.active_start_date ? new Date(formData.active_start_date) : undefined,
												to: formData.active_end_date ? new Date(formData.active_end_date) : undefined,
											}}
											onChange={(date) => {
												setFormData((prev) => ({
													...prev,
													active_start_date: date?.from ? date.from.toISOString(): "",
													active_end_date: date?.to ? date.to.toISOString() : "",
												}));
											}}
										/>
									</div>
								</div>
							</div>
							<UserRole userId={id!} />
							<div>
								<Button type="submit" className="w-full" disabled={loading}>
									{loading ? "Уншиж байна..." : id ? "Засах" : "Нэмэх"}
								</Button>
							</div>
						</form>
					</CardContent>
				</Card>
			</div>
		);
	}
	
\end{lstlisting}
\pagebreak
\section{Back-End}

\subsection{/Controller/Admin/userController.go}
\begin{lstlisting}[language=Go, frame=single]
	package controllers

	import (
		"branch-admin-service/log"
		"branch-admin-service/models"
		services "branch-admin-service/services/admin"
		"net/http"
	
		"github.com/gin-gonic/gin"
	)
	
	func GetAllUsers(c *gin.Context) {
		id := c.Query("id")
	
		if id != "" {
			user, err := services.GetUserByID(id, c)
			if err != nil {
				log.Error("Failed to fetch user by ID", map[string]interface{}{"error": err.Error(), "id": id}, c)
				c.JSON(http.StatusInternalServerError, gin.H{"error": "Failed to fetch user"})
				return
			}
			if user == nil {
				log.Warn("User not found", map[string]interface{}{"id": id}, c)
				c.JSON(http.StatusNotFound, gin.H{"error": "User not found"})
				return
			}
			log.Info("User fetched successfully", map[string]interface{}{"id": id}, c)
			c.JSON(http.StatusOK, user)
			return
		}
	
		users, err := services.GetAllUsers(c)
		if err != nil {
			log.Error("Failed to fetch users", map[string]interface{}{"error": err.Error()}, c)
			c.JSON(http.StatusInternalServerError, gin.H{"error": "Failed to fetch users"})
			return
		}
		log.Info("Users fetched successfully", map[string]interface{}{"count": len(users)}, c)
		c.JSON(http.StatusOK, users)
	}
	
	func CreateUser(c *gin.Context) {
		var user models.User
		if err := c.ShouldBindJSON(&user); err != nil {
			log.Error("Failed to bind user JSON", map[string]interface{}{"error": err.Error()}, c)
			c.JSON(http.StatusBadRequest, gin.H{"error": "Invalid request payload"})
			return
		}
	
		log.Info("Creating user", map[string]interface{}{"user": user}, c)
	
		newUser, err := services.CreateUser(user, c.Request)
		if err != nil {
			log.Error("Failed to create user", map[string]interface{}{"user": user, "error": err.Error()}, c)
			c.JSON(http.StatusInternalServerError, gin.H{"error": err.Error()})
			return
		}
	
		log.Info("User created successfully", map[string]interface{}{"userID": newUser.ID}, c)
		c.JSON(http.StatusCreated, newUser)
	}
	
	func UpdateUser(c *gin.Context) {
		id := c.Param("id")
		var user models.User
	
		if err := c.ShouldBindJSON(&user); err != nil {
			log.Error("Failed to bind user JSON", map[string]interface{}{"error": err.Error()}, c)
			c.JSON(http.StatusBadRequest, gin.H{"error": "Invalid request payload"})
			return
		}
	
		log.Info("Updating user", map[string]interface{}{"userID": id, "user": user}, c)
		updatedUser, err := services.UpdateUser(id, user, c.Request)
		if err != nil {
			log.Error("Failed to update user", map[string]interface{}{"userID": id, "error": err.Error()}, c)
			c.JSON(http.StatusInternalServerError, gin.H{"error": "Failed to update user"})
			return
		}
	
		log.Info("User updated successfully", map[string]interface{}{"userID": updatedUser.ID}, c)
		c.JSON(http.StatusOK, updatedUser)
	}
	
	func DeleteUser(c *gin.Context) {
		id := c.Param("id")
	
		if err := services.DeleteUser(id, c.Request); err != nil {
			log.Error("Failed to delete user", map[string]interface{}{"userID": id, "error": err.Error()}, c)
			c.JSON(http.StatusInternalServerError, gin.H{"error": "Failed to delete user"})
			return
		}
	
		log.Info("User deleted successfully", map[string]interface{}{"userID": id}, c)
		c.JSON(http.StatusOK, gin.H{"message": "User deleted successfully"})
	}
	
	func CheckLoginIDExists(c *gin.Context) {
		loginID := c.Query("login_id")
		if loginID == "" {
			log.Error("login_id is required", map[string]interface{}{}, c)
			c.JSON(http.StatusBadRequest, gin.H{"error": "login_id is required"})
			return
		}
	
		exists, err := services.CheckLoginIDExists(loginID, c)
		if err != nil {
			log.Error("Failed to check login_id", map[string]interface{}{"error": err.Error()}, c)
			c.JSON(http.StatusInternalServerError, gin.H{"error": "Failed to check login_id"})
			return
		}
	
		log.Info("Checked login_id existence", map[string]interface{}{"login_id": loginID, "exists": exists}, c)
		c.JSON(http.StatusOK, gin.H{"exists": exists})
	}
	
\end{lstlisting}

\end{document}
