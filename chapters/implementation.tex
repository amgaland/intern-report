%----------------------------------------------------------------------------------------
%  GORM Хэрэгжүүлэлт
%----------------------------------------------------------------------------------------

\section{GORM хэрэгжүүлэлт}

Уг функцуудыг шаардлагын дагуу алхам алхмаар хэрэгжүүлсэнээр GORM-ийг ашиглах анхан шатны туршлагатай болох зорилготой байсан ба цаашид ажилласан төсөл дээр GORM-ийн дотоод логик, өгөгдлийн сангийн холболт, тайлбар хийх, өгөгдөл хадгалах болон шинэчлэх процессуудын ажиллагааг ойлгох, тодорхой хэмжээнд практик мэдлэгийг цуглуулж чадсан.

Сурах ур чадвар: 
\begin{itemize} 
	\item GORM-ийг ашиглах чадвараа нэмэгдүүлэх 
	\item Өгөгдлийн сантай харьцах үйлдлүүд (GET, POST, UPDATE, DELETE) болон тэдгээрийн процессийг ойлгох 
	\item Хэрэглэгчийн логийг бүртгэх болон алдааг барих чадвар эзэмших 
	\item CRUD үйлдлүүдийг GORM ашиглан бичих 
	\item Горм-ийг ашиглахдаа data validation болон password hashing-ийн талаар мэдлэгтэй болох \end{itemize}

Функцийн шаардлага: 
\begin{itemize} 
	\item GORM ашиглан хэрэглэгчийн мэдээллийг авах, нэмэх, засах, устгах 
	\item Хэрэглэгчийн нэвтрэх ID-г шалгах функц бичих 
	\item Алдааг бүртгэх болон лог хийх функц ашиглах 
	\item GORM-ийн дотоод функцууд болон query үйлдлийг ойлгох 
	\item Хэрэглэгчийн мэдээлэл устгах үед хамааралтай мэдээллийг устгах
\end{itemize}
	
\subsection{User Management Functions}

Энэхүү хэсэгт хэрэглэгчийн мэдээллийг удирдах хэд хэдэн функцийг оруулсан болно. Үүнд хэрэглэгчийн мэдээллийг авах, шинэчлэх, устгах, болон логины ID-г шалгах үйлдлүүд орно.

\begin{lstlisting}[language=Go, caption=Get all users function, frame=single]
func GetAllUsers(r *http.Request) ([]models.User, error) {
	var users []models.User
	if err := config.DB.Find(&users).Error; err != nil {
		log.Error("Failed to fetch users", map[string]interface{}{"error": err.Error()}, r)
		return nil, err
	}
	return users, nil
}
\end{lstlisting}

Энэ функц нь бүх хэрэглэгчдийн мэдээллийг өгөгдлийн сангаас авч, ямар нэгэн алдаа гарсан тохиолдолд тухайн алдааг логлош болно.

\begin{lstlisting}[language=Go, caption=Create user function, frame=single]
func CreateUser(user models.User, r *http.Request) (models.User, error) {
	existingUser := config.DB.Where("login_id = ?", user.LoginID).First(&models.User{})
	if existingUser.RowsAffected > 0 {
		return models.User{}, errors.New("user already exists")
	}

	hashedPassword, err := utils.HashPassword(user.Password)
	if err != nil {
		return models.User{}, err
	}

	user.Password = hashedPassword

	if err := config.DB.Create(&user).Error; err != nil {
		return models.User{}, err
	}
	return user, nil
}
\end{lstlisting}

Шинэ хэрэглэгчийг үүсгэх үед, хэрэглэгчийн ID өмнө нь байгаа эсэхийг шалгаад, байсан бол алдаа буцаана. Мөн хэрэглэгчийн нууц үгийг хуулаад өгөгдлийн сан дээр хадгална.

\begin{lstlisting}[language=Go, caption=Update user function, frame=single]
func UpdateUser(id string, user models.User, r *http.Request) (models.User, error) {
	hashedPassword, err := utils.HashPassword(user.Password)
	if err != nil {
		return models.User{}, err
	}

	user.Password = hashedPassword
	if err := config.DB.Model(user).Where("id = ?", id).Updates(user).Error; err != nil {
		return models.User{}, err
	}
	return user, nil
}
\end{lstlisting}

Энэ функц нь хэрэглэгчийн мэдээллийг шинэчилж, нууц үгийг шинэчлэх шаардлагатай бол шинэчилнэ.

\begin{lstlisting}[language=Go, caption=Delete user function, frame=single]
func DeleteUser(id string, r *http.Request) error {
	if err := config.DB.Where("user_id = ?", id).Delete(&models.UserRole{}).Error; err != nil {
		return err
	}

	if err := config.DB.Where("id = ?", id).Delete(&models.User{}).Error; err != nil {
		return err
	}

	return nil
}
\end{lstlisting}

Хэрэглэгчийг устгах үед түүний хэрэглэгчийн үүрэг болон хэрэглэгчийн мэдээллийг устгана.

\begin{lstlisting}[language=Go, caption=Get user by ID function, frame=single]
func GetUserByID(id string, r *http.Request) (*models.User, error) {
	log.Info("Fetching user by ID from the database", map[string]interface{}{"id": id}, r)

	var user models.User

	err := config.DB.Where("id = ?", id).First(&user).Error
	if err != nil {
		if errors.Is(err, gorm.ErrRecordNotFound) {
			log.Warn("User not found", map[string]interface{}{"id": id}, r)
			return nil, nil
		}
		log.Error("Database query error", map[string]interface{}{"error": err.Error()}, r)
		return nil, err
	}

	log.Info("User fetched successfully", map[string]interface{}{"id": id}, r)
	return &user, nil
}
\end{lstlisting}

Энэхүү функц нь хэрэглэгчийн ID-г ашиглан мэдээллийг өгөгдлийн сангаас авдаг. Хэрэв хэрэглэгч олдвол түүний мэдээллийг буцаана.

\begin{lstlisting}[language=Go, caption=Check if login ID exists function, frame=single]
func CheckLoginIDExists(loginID string, r *http.Request) (bool, error) {
	var user models.User
	result := config.DB.Where("login_id = ?", loginID).First(&user)

	if result.Error != nil {
		if errors.Is(result.Error, gorm.ErrRecordNotFound) {
			return false, nil
		}
		log.Error("Failed to check login_id", map[string]interface{}{"error": result.Error.Error()}, r)
		return false, result.Error 
	}

	return true, nil 
}
\end{lstlisting}

Энэ функц нь өгөгдлийн сан дээрх login ID-гийн байршлыг шалгадаг. Хэрэв байвал true, байхгүй бол false буцаана.
\pagebreak

%----------------------------------------------------------------------------------------
%  Golang test хэрэгжүүлэлт
%----------------------------------------------------------------------------------------


\section{Golang Test хэрэгжүүлэлт}

Энэхүү хэсэгт бид Go хэл дээрх үйлдлүүдийг тест хийх зориулалттай функцуудыг хэрэгжүүлсэн болно. Тестүүд нь хэрэглэгчийн мэдээллийг нэмэх, авах, шинэчлэх, устгах зэрэг CRUD үйлдлүүдийг шалгана.

\subsection{Тестийн орчин}

Тестийн орчин байгуулахдаа өгөгдлийн санг анхны нөхцөлд нь оруулж, дараа нь алдаагүйгээр бүх функцуудыг туршина. Энэ зорилгоор дараах функцүүдийг ашиглав:

\begin{lstlisting}[language=Go, caption=Test Database Setup, frame=single]
func setupTestDB() *gorm.DB {
    test.ConnectTestDatabase()
    config.DB.Exec("TRUNCATE TABLE users RESTART IDENTITY CASCADE") 
    return config.DB
}
\end{lstlisting}

Энэхүү функц нь тестийн өгөгдлийн санг тохируулж, хэрэглэгчийн хүснэгтийг шинэчилнэ.

\subsection{Тестийн функцууд}

\begin{itemize}
    \item \texttt{TestCreateUser}: Шинэ хэрэглэгч үүсгэх үйлдлийг шалгана.
    \item \texttt{TestGetAllUsers}: Бүх хэрэглэгчийн мэдээллийг авах функцыг туршина.
    \item \texttt{TestUpdateUser}: Хэрэглэгчийн мэдээллийг шинэчлэх үйлдлийг шалгана.
    \item \texttt{TestDeleteUser}: Хэрэглэгчийн мэдээллийг устгах үйлдлийг туршина.
    \item \texttt{TestGetUserByID}: Хэрэглэгчийн ID-ээр мэдээлэл авах үйлдлийг шалгана.
\end{itemize}

\subsubsection{TestCreateUser}

Энэхүү тест нь шинэ хэрэглэгчийг үүсгэж, үүсгэх үед гарах алдааг шалгах бөгөөд хэрэглэгчийн нууц үгийг хашлах явцыг шалгана.

\begin{lstlisting}[language=Go, caption=Test Create User Function, frame=single]
func TestCreateUser(t *testing.T) {
    db := setupTestDB()
    defer test.CloseTestDatabase()

    loginID := "johndoe"
    dropExistingUser(db, loginID)

    r := httptest.NewRequest(http.MethodPost, "/", nil)
    user := createSampleUser(loginID)

    createdUser, err := services.CreateUser(user, r)

    assert.NoError(t, err, "Error creating user")
    assert.NotZero(t, createdUser.ID, "User ID should not be zero")
    assert.NotEqual(t, "password123", createdUser.Password, "Password should be hashed")

    var fetchedUser models.User
    err = db.First(&fetchedUser, "id = ?", createdUser.ID).Error
    assert.NoError(t, err, "Error fetching user")

    assert.Equal(t, "John", fetchedUser.FirstName, "First name mismatch")
    assert.Equal(t, "Doe", fetchedUser.LastName, "Last name mismatch")
    assert.NotEqual(t, "password123", fetchedUser.Password, "Password should be hashed")
}
\end{lstlisting}

\subsubsection{TestGetAllUsers}

Энэхүү тест нь өгөгдлийн сан дахь бүх хэрэглэгчийн мэдээллийг авах үйлдлийг шалгана. Мөн хэрэглэгчдийн тоог болон мэдээллийг баталгаажуулна.

\begin{lstlisting}[language=Go, caption=Test Get All Users Function, frame=single]
func TestGetAllUsers(t *testing.T) {
    db := setupTestDB()
    defer test.CloseTestDatabase()

    db.Exec("TRUNCATE TABLE users RESTART IDENTITY CASCADE")

    users := []models.User{
        {FirstName: "John", LastName: "Doe"},
        {FirstName: "Jane", LastName: "Smith"},
    }
    for _, user := range users {
        db.Create(&user)
    }

    r := httptest.NewRequest(http.MethodGet, "/users", nil)
    allUsers, err := services.GetAllUsers(r)

    assert.NoError(t, err, "Error fetching all users")
    assert.Len(t, allUsers, len(users), "Unexpected number of users fetched")

    for i, fetchedUser := range allUsers {
        assert.True(t, i < len(users), "Index %d out of range for expected users", i)
        assert.Equal(t, users[i].FirstName, fetchedUser.FirstName, 
            "First name mismatch. Expected: %s, Got: %s", users[i].FirstName, fetchedUser.FirstName)
        assert.Equal(t, users[i].LastName, fetchedUser.LastName, 
            "Last name mismatch. Expected: %s, Got: %s", users[i].LastName, fetchedUser.LastName)
    }
}
\end{lstlisting}

\subsubsection{TestUpdateUser}

Энэхүү тест нь хэрэглэгчийн мэдээллийг шинэчлэх үйлдлийг шалгадаг.

\begin{lstlisting}[language=Go, caption=Test Update User Function, frame=single]
func TestUpdateUser(t *testing.T) {
    db := setupTestDB()
    defer test.CloseTestDatabase()

    // Create sample user
    originalUser := createSampleUser("johndoe")
    db.Create(&originalUser)

    // Prepare updated user data
    updatedUserData := models.User{
        FirstName: "Jane",
        LastName:  "Smith",
        Password:  "newpassword123",
    }

    r := httptest.NewRequest(http.MethodPut, "/user", nil)
    updatedUser, err := services.UpdateUser(originalUser.ID, updatedUserData, r)

    assert.NoError(t, err, "Error updating user")
    assert.Equal(t, updatedUserData.FirstName, updatedUser.FirstName, "First name mismatch")
    assert.Equal(t, updatedUserData.LastName, updatedUser.LastName, "Last name mismatch")
    assert.NotEqual(t, "newpassword123", updatedUser.Password, "Password should be hashed")

    var fetchedUser models.User
    err = db.First(&fetchedUser, "id = ?", originalUser.ID).Error
    assert.NoError(t, err, "Error fetching updated user")

    assert.Equal(t, updatedUserData.FirstName, fetchedUser.FirstName, "First name mismatch in database")
    assert.Equal(t, updatedUserData.LastName, fetchedUser.LastName, "Last name mismatch in database")
    assert.NotEqual(t, "newpassword123", fetchedUser.Password, "Password should be hashed in database")
}
\end{lstlisting}
\pagebreak

%----------------------------------------------------------------------------------------
%  Front-End хэрэгжүүлэлт
%---------------------------------------------------------------------------------------
\section{Front-End хэрэгжүүлэлт}
энд юм бичих

Сурах ур чадвар:
\begin{itemize}
	\item Next.js -ийн талаар үндсэн ойлголт авах
	\item API-тай харьцах
	\item Формыг хянаж алдаатай тохиолдолд хэрэглэгчид мэдэгдэх
	\item UI/UX сайжруулалт
	\item State Management ба өгөгдлийн хөрвүүлэлт хувийн мэдээлэл болон огноо зэргийг хэрэглэгчид ойлгогдохуйц хөрвүүлэх, хадгалах
	\item Аюулгүй байдал, баталгаажуулалт token ашиглан хамгаалалттай API хүсэлт илгээх
\end{itemize}

Формын шаардлага:
\begin{itemize}
	\item Next.js ашигласан байх
	\item Хэрэглэгчийн мэдээлэл удирдах
	\item Форм хэлбэртэй байх
	\item API хүсэлт илгээдэг байх
	\item Токен ашиглан хамгаалалттай өгөгдөл илгээх
	\item Хэрэглэгчийн үүргийг тохируулах
\end{itemize}
\pagebreak

\subsection{Front-end-ийн ерөнхий бүх элементийг агуулсан хэсэг болох формд засвар хийх код}

Хэрэглэгчийн мэдээллийг хадгалах интерфейс

\begin{lstlisting}[language=Typescript, caption=UserFormData интерфейсийг үүсгэсэн байдал, frame=single]
interface UserFormData {
  first_name: string;
  last_name: string;
  login_id: string;
  email_work: string;
  email_personal: string;
  phone_number_work: string;
  phone_number_personal: string;
  is_active: boolean;
  active_start_date: string;
  active_end_date: string;
  password?: string;
  created_by: string;
  updated_by: string;
}
\end{lstlisting}

Энэ функц нь хэрэглэгчийн мэдээллийг нэмэх, засах формыг харуулна
\begin{lstlisting}[language=Typescript, caption=UserFormPage функц (Main component), frame=single]
export default function UserFormPage() {}
\end{lstlisting}

Доорх функууд UserFormPage дотор агуулагдах функууд. Эхлээд хувьсагчаа тодорхойлж өгнө.
\begin{lstlisting}[language=Typescript, caption=Хувьсагч, frame=single]
	const searchParams = useSearchParams();
  const id = searchParams.get("id");
  const router = useRouter();
  const { data: session } = useSession();
\end{lstlisting}

UserFormData тохируулж өгнө. Created_by болон updated_by нь нэвтэрсэн хэрэглэгчийн id-гаар автоматаар бөглөгдөнө.

\begin{lstlisting}[language=Typescript, caption=initialFormData хоосон утга оноосон байдал, frame=single]
	const initialFormData: UserFormData = {
    first_name: "",
    last_name: "",
    login_id: "",
    email_work: "",
    email_personal: "",
    phone_number_work: "",
    phone_number_personal: "",
    is_active: false,
    active_start_date: "",
    active_end_date: "",
    password: "",
    created_by: session?.user?.id || "",
    updated_by: session?.user?.id || "",
  };
\end{lstlisting}

Төлөв хадгалах
\begin{lstlisting}[language=Typescript, caption=Төлөв хадгалах, frame=single]
	const [formData, setFormData] = useState<UserFormData>(initialFormData);
  const [loading, setLoading] = useState(false);
  const [errors, setErrors] = useState<Record<string, string>>({});
\end{lstlisting}


Хэрэглэгчийн мэдээллийг татах
\begin{lstlisting}[language=Typescript, caption=Хэрэглэгчийн мэдээлэл татах useEffect, frame=single]
	useEffect(() => {
    const fetchUserData = async () => {
      if (!id || !session?.user?.token) return;

      try {
        setLoading(true);
        const response = await req.GET(
          `/admin/users?id=${id}`,
          session.user.token
        );
        setFormData({
          ...response,
          updated_by: session.user.id || "",
        });
      } catch (error) {
        toast({
          title: "Алдаа",
          description: "Хэрэглэгчийн мэдээлэл авахад алдаа гарлаа",
          variant: "destructive",
        });
      } finally {
        setLoading(false);
      }
    };

    fetchUserData();
  }, [id, session?.user?.id, session?.user?.token]);
\end{lstlisting}

Оролтын өөрчлөлт хянах handleChange функц
\begin{lstlisting}[language=Typescript, caption=handleChange, frame=single]
	const handleChange = (e: React.ChangeEvent<HTMLInputElement>) => {
    const { name, value, type, checked } = e.target;
    setFormData((prev) => ({
      ...prev,
      [name]: type === "checkbox" ? checked : value,
    }));
  };
\end{lstlisting}

Хүсэлт илгээх handleSubmit функц
\begin{lstlisting}[language=Typescript, caption=handleSubmit, frame=single]
	const handleSubmit = async (e: React.FormEvent) => {
    e.preventDefault();

    try {
      setLoading(true);
      const response = await req.PUT(
        `/admin/users/${id || ""}`,
        session?.user?.token || "",
        {
          ...formData,
          active_start_date: formData.active_start_date || null,
          active_end_date: formData.active_end_date || null,
        }
      );

      if (response) {
        toast({
          title: "Амжилттай",
          description: "Мэдээлэл шинэчлэгдлээ",
        });
      }
    } catch (error) {
      console.error("Error submitting form:", error);
      toast({
        title: "Алдаа",
        description: "Гэнэтийн алдаа гарлаа",
        variant: "destructive",
      });
    } finally {
      setLoading(false);
    }
  };
\end{lstlisting}

Формын UI бүтэц
\begin{lstlisting}[language=Typescript, caption=UI бүтэц, frame=single]
	return (
    <div className="py-8 px-4">
      <Card>
        <CardHeader>
          <CardTitle className="text-2xl">
            {id ? "Хэрэглэгч засах" : "Шинэ хэрэглэгч нэмэх"}
          </CardTitle>
        </CardHeader>
        <CardContent>
          <form onSubmit={handleSubmit} className="space-y-4">
            <div className="space-y-6">
              <h3 className="text-lg font-medium">Хувийн мэдээлэл</h3>
              <div className="grid grid-cols-1 md:grid-cols-2 gap-4">
                <div>
                  <Label htmlFor="first_name">Нэр</Label>
                  <Input
                    id="first_name"
                    name="first_name"
                    type="text"
                    value={formData.first_name}
                    onChange={handleChange}
                    className={errors.first_name ? "border-red-500" : ""}
                  />
                  {errors.first_name && (
                    <p className="text-red-500 text-sm mt-1">
                      {errors.first_name}
                    </p>
                  )}
                </div>
                <div>
                  <Label htmlFor="last_name">Овог</Label>
                  <Input
                    id="last_name"
                    name="last_name"
                    type="text"
                    value={formData.last_name}
                    onChange={handleChange}
                    className={errors.last_name ? "border-red-500" : ""}
                  />
                  {errors.last_name && (
                    <p className="text-red-500 text-sm mt-1">
                      {errors.last_name}
                    </p>
                  )}
                </div>
                <div>
                  <Label htmlFor="login_id">Нэвтрэх ID</Label>
                  <Input
                    id="login_id"
                    name="login_id"
                    type="text"
                    value={formData.login_id}
                    onChange={handleChange}
                    className={errors.login_id ? "border-red-500" : ""}
                  />
                  {errors.login_id && (
                    <p className="text-red-500 text-sm mt-1">
                      {errors.login_id}
                    </p>
                  )}
                </div>
                {!id && (
                  <div>
                    <Label htmlFor="password">Нууц үг</Label>
                    <Input
                      id="password"
                      name="password"
                      type="password"
                      value={formData.password}
                      onChange={handleChange}
                      className={errors.password ? "border-red-500" : ""}
                    />
                    {errors.password && (
                      <p className="text-red-500 text-sm mt-1">
                        {errors.password}
                      </p>
                    )}
                  </div>
                )}
              </div>
            </div>
            <div className="space-y-2">
              <h3 className="text-lg font-medium">Холбоо барих</h3>
              <div className="grid grid-cols-1 md:grid-cols-2 gap-4">
                <div>
                  <Label htmlFor="email_work">Ажлын имэйл</Label>
                  <Input
                    id="email_work"
                    name="email_work"
                    type="email"
                    value={formData.email_work}
                    onChange={handleChange}
                    className={errors.email_work ? "border-red-500" : ""}
                  />
                  {errors.email_work && (
                    <p className="text-red-500 text-sm mt-1">
                      {errors.email_work}
                    </p>
                  )}
                </div>
                <div>
                  <Label htmlFor="email_personal">Хувийн имэйл</Label>
                  <Input
                    id="email_personal"
                    name="email_personal"
                    type="email"
                    value={formData.email_personal}
                    onChange={handleChange}
                    className={errors.email_personal ? "border-red-500" : ""}
                  />
                  {errors.email_personal && (
                    <p className="text-red-500 text-sm mt-1">
                      {errors.email_personal}
                    </p>
                  )}
                </div>
                <div>
                  <Label htmlFor="phone_number_work">Ажлын дугаар</Label>
                  <Input
                    id="phone_number_work"
                    name="phone_number_work"
                    type="tel"
                    value={formData.phone_number_work}
                    onChange={handleChange}
                  />
                </div>
                <div>
                  <Label htmlFor="phone_number_personal">Хувийн дугаар</Label>
                  <Input
                    id="phone_number_personal"
                    name="phone_number_personal"
                    type="tel"
                    value={formData.phone_number_personal}
                    onChange={handleChange}
                  />
                </div>
              </div>
            </div>
            <div className="space-y-2">
              <h3 className="text-lg font-medium">Төлөв</h3>
              <div className="grid grid-cols-1 md:grid-cols-2 gap-4 items-center">
                <div className="flex items-center space-x-2">
                  <input
                    id="is_active"
                    name="is_active"
                    type="checkbox"
                    checked={formData.is_active}
                    onChange={handleChange}
                    className="h-4 w-4 border-gray-300 rounded"
                  />
                  <Label htmlFor="is_active">Идэвхтэй эсэх</Label>
                </div>
                <div className="space-x-2">
                  <Label>Ажиллах хугацаа:</Label>
                  <DatePickerWithRange
                    value={{
                      from: formData.active_start_date ? new Date(formData.active_start_date) : undefined,
                      to: formData.active_end_date ? new Date(formData.active_end_date) : undefined,
                    }}
                    onChange={(date) => {
                      setFormData((prev) => ({
                        ...prev,
                        active_start_date: date?.from ? date.from.toISOString(): "",
                        active_end_date: date?.to ? date.to.toISOString() : "",
                      }));
                    }}
                  />
                </div>
              </div>
            </div>
            <UserRole userId={id!} />
            <div>
              <Button type="submit" className="w-full" disabled={loading}>
                {loading ? "Уншиж байна..." : id ? "Засах" : "Нэмэх"}
              </Button>
            </div>
          </form>
        </CardContent>
      </Card>
    </div>
  );
\end{lstlisting}

%----------------------------------------------------------------------------------------
%  Back-End хэрэгжүүлэлт
%----------------------------------------------------------------------------------------

