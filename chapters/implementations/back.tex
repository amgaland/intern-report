\section{Back-End хэрэгжүүлэлт}
Энэ хэсэгт хэрэглэгчийн мэдээлэлтэй холбоотой CRUD үйлдлүүдийг GORM ашиглан хийнэ.
CRUD(create, Read, Update, Delete) үйлдлүүд нь хэрэглэгчийн мэдээллийг үүсгэх, унших,
шинэчлэх, устгах боломжийг олгодог.

Сурах ур чадвар:
\begin{itemize}
	\item Golang хэлний мэдлэг
	\item Gin framework ашиглах
	\item GORM ORM ашиглах
	\item Өгөгдлийн сангийн зохицуулалт хийх
	\item Hashing ба нууцлал
	\item Алдаа удирдах
	\item Лог хөтлөх
	\item HTTP хүсэлттэй ажилллах
	\item Кодын бүтэц зохион байгуулалт
\end{itemize}

Кодны шаардлага:
\begin{itemize}
	\item Golang хэлний синтакс, өгөгдлийн төрөл, функцийн ажиллах зарчим мэдэх 
	\item REST API зохион байгуулах, маршрутын менежмент хийх чадвартай байх.
	\item GORM ашиглан өгөгдлийн сангийн хүсэлтүүд боловсруулах чадвартай байх.
	\item CRUD үйлдлүүдийг бүтээх
	\item JSON өгөгдлийг боловсруулах
	\item Нууц үгийн хашлалат, хэрэглэгчийн мэдээллийг хамгаалах
	\item API-д гарч болох алдааг зөв удирдаж, тохирсон HTTP код буцаах чадвартай байх
	\item Controller  давхаргыг зөв зохион байгуулах
	\item API тест хийх
\end{itemize}
\pagebreak

\subsection{GetAllUsers}
Хэрэглэгчийн мэдээллийг Postgres өгөгдлийн сангаас авах API-ийг хэрэгжүүлэх хэсэг. GetAllUsers функц нь өгөгдлийн сангаас бүх хэрэглэгчийн мэдээллийг авахад зориулагдсан бөгөөд хэрэглэгчийн жагсаалтыг буцаадаг.

\begin{lstlisting}[language=Go, caption=CreateUser Controller, frame=single]
	func CreateUser(c *gin.Context) {
	var user models.User
	if err := c.ShouldBindJSON(&user); err != nil {
		log.Error("Failed to bind user JSON", map[string]interface{}{"error": err.Error()}, c)
		c.JSON(http.StatusBadRequest, gin.H{"error": "Invalid request payload"})
		return
	}

	log.Info("Creating user", map[string]interface{}{"user": user}, c)

	newUser, err := services.CreateUser(user, c.Request)
	if err != nil {
		log.Error("Failed to create user", map[string]interface{}{"user": user, "error": err.Error()}, c)
		c.JSON(http.StatusInternalServerError, gin.H{"error": err.Error()})
		return
	}

	log.Info("User created successfully", map[string]interface{}{"userID": newUser.ID}, c)
	c.JSON(http.StatusCreated, newUser)
}
\end{lstlisting}

\subsection{CreateUser}
CreateUser функц нь хэрэглэгчийн мэдээллийг JSON хэлбэрээр хүлээн авч, models.User объект болгон хувиргадаг. Ингэснээр хэрэглэгчийн мэдээллийг сервер рүү илгээх боломжтой болно.
\begin{lstlisting}[language=Go, caption=CreatUser Controller, frame=single]
	func GetAllUsers(r *http.Request) ([]models.User, error) {
		var users []models.User
		if err := config.DB.Find(&users).Error; err != nil {
			log.Error("Failed to fetch users", map[string]interface{}{"error": err.Error()}, r)
			return nil, err
		}
		return users, nil
	}
\end{lstlisting}

\subsection{UpdateUser}
UpdateUser функц нь хэрэглэгчийн ID болон шинэ мэдээллийг хүлээн авч, models.User объектоор шинэчилнэ.
\begin{lstlisting}[language=Go, caption=UpdateUser Controller, frame=single]
	func UpdateUser(c *gin.Context) {
	id := c.Param("id")
	var user models.User

	if err := c.ShouldBindJSON(&user); err != nil {
		log.Error("Failed to bind user JSON", map[string]interface{}{"error": err.Error()}, c)
		c.JSON(http.StatusBadRequest, gin.H{"error": "Invalid request payload"})
		return
	}

	log.Info("Updating user", map[string]interface{}{"userID": id, "user": user}, c)
	updatedUser, err := services.UpdateUser(id, user, c.Request)
	if err != nil {
		log.Error("Failed to update user", map[string]interface{}{"userID": id, "error": err.Error()}, c)
		c.JSON(http.StatusInternalServerError, gin.H{"error": "Failed to update user"})
		return
	}

	log.Info("User updated successfully", map[string]interface{}{"userID": updatedUser.ID}, c)
	c.JSON(http.StatusOK, updatedUser)
}
\end{lstlisting}

\subsection{DeleteUser}
DeleteUser функц нь хэрэглэгчийн ID-ийг URL параметрээс авч, тухайн хэрэглэгчийг устгах үйлдлийг гүйцэтгэдэг.
\begin{lstlisting}[language=Go, caption=DeleteUser Controller, frame=single]
	func DeleteUser(c *gin.Context) {
		id := c.Param("id")

		if err := services.DeleteUser(id, c.Request); err != nil {
			log.Error("Failed to delete user", map[string]interface{}{"userID": id, "error": err.Error()}, c)
			c.JSON(http.StatusInternalServerError, gin.H{"error": "Failed to delete user"})
			return
		}

		log.Info("User deleted successfully", map[string]interface{}{"userID": id}, c)
		c.JSON(http.StatusOK, gin.H{"message": "User deleted successfully"})
	}
\end{lstlisting}

\subsection{CheckLoginIDExists}
CheckLoginIDExists функц нь хэрэглэгчийн login id-ийг URL-ийн query параметрээс авч, өгөгдлийн санд login id-ийн оршин байгаа эсэхийг шалгадаг.
\begin{lstlisting}[language=Go, caption=CheckLoginIDExists Controller, frame=single]
func CheckLoginIDExists(c *gin.Context) {
	loginID := c.Query("login_id")
	if loginID == "" {
		log.Error("login_id is required", map[string]interface{}{}, c)
		c.JSON(http.StatusBadRequest, gin.H{"error": "login_id is required"})
		return
	}

	exists, err := services.CheckLoginIDExists(loginID, c)
	if err != nil {
		log.Error("Failed to check login_id", map[string]interface{}{"error": err.Error()}, c)
		c.JSON(http.StatusInternalServerError, gin.H{"error": "Failed to check login_id"})
		return
	}

	log.Info("Checked login_id existence", map[string]interface{}{"login_id": loginID, "exists": exists}, c)
	c.JSON(http.StatusOK, gin.H{"exists": exists})
}

\end{lstlisting}
