\section{Front-End хэрэгжүүлэлт}
Энэхүү төсөл нь Next.js фреймворкийг ашиглан хэрэглэгчийн бүртгэлд засвар хийх хуудас үүсгэсэн хэсгийг харуулж байна. Хэрэглэгчийн өгөгдлийг хүлээн авч, сервер рүү илгээх, мөн серверээс хариу авах процессыг оролцуулан бичигдсэн. Үндсэн зорилго нь хэрэглэгчийн бүртгэл болон мэдээллийн санах оруулах үндсэн үйл ажиллагааг вэб хуудас дээр хэрэгжүүлэх явдал юм.

Сурах ур чадвар:
\begin{itemize}
	\item Next.js -ийн талаар үндсэн ойлголт авах
	\item API-тай харьцах
	\item Формыг хянаж алдаатай тохиолдолд хэрэглэгчид мэдэгдэх
	\item UI/UX сайжруулалт
	\item State Management ба өгөгдлийн хөрвүүлэлт хувийн мэдээлэл болон огноо зэргийг хэрэглэгчид ойлгогдохуйц хөрвүүлэх, хадгалах
	\item Аюулгүй байдал, баталгаажуулалт token ашиглан хамгаалалттай API хүсэлт илгээх
\end{itemize}

Формын шаардлага:
\begin{itemize}
	\item Next.js ашигласан байх
	\item Хэрэглэгчийн мэдээлэл удирдах
	\item Форм хэлбэртэй байх
	\item API хүсэлт илгээдэг байх
	\item Токен ашиглан хамгаалалттай өгөгдөл илгээх
	\item Хэрэглэгчийн үүргийг тохируулах
\end{itemize}
\pagebreak

\subsection{Front-end-ийн ерөнхий бүх элементийг агуулсан хэсэг болох формд засвар хийх код}

Хэрэглэгчийн мэдээллийг хадгалах интерфейс

\begin{lstlisting}[language=Typescript, caption=UserFormData интерфейсийг үүсгэсэн байдал, frame=single]
interface UserFormData {
  first_name: string;
  last_name: string;
  login_id: string;
  email_work: string;
  email_personal: string;
  phone_number_work: string;
  phone_number_personal: string;
  is_active: boolean;
  active_start_date: string;
  active_end_date: string;
  password?: string;
  created_by: string;
  updated_by: string;
}
\end{lstlisting}

Энэ функц нь хэрэглэгчийн мэдээллийг нэмэх, засах формыг харуулна
\begin{lstlisting}[language=Typescript, caption=UserFormPage функц (Main component), frame=single]
export default function UserFormPage() {}
\end{lstlisting}

Доорх функууд UserFormPage дотор агуулагдах функууд. Эхлээд хувьсагчаа тодорхойлж өгнө.
\begin{lstlisting}[language=Typescript, caption=Хувьсагч, frame=single]
	const searchParams = useSearchParams();
  const id = searchParams.get("id");
  const router = useRouter();
  const { data: session } = useSession();
\end{lstlisting}

UserFormData тохируулж өгнө. Created by болон updated by нь нэвтэрсэн хэрэглэгчийн id-гаар автоматаар бөглөгдөнө.

\begin{lstlisting}[language=Typescript, caption=initialFormData хоосон утга оноосон байдал, frame=single]
	const initialFormData: UserFormData = {
    first_name: "",
    last_name: "",
    login_id: "",
    email_work: "",
    email_personal: "",
    phone_number_work: "",
    phone_number_personal: "",
    is_active: false,
    active_start_date: "",
    active_end_date: "",
    password: "",
    created_by: session?.user?.id || "",
    updated_by: session?.user?.id || "",
  };
\end{lstlisting}

Төлөв хадгалах
\begin{lstlisting}[language=Typescript, caption=Төлөв хадгалах, frame=single]
	const [formData, setFormData] = useState<UserFormData>(initialFormData);
  const [loading, setLoading] = useState(false);
  const [errors, setErrors] = useState<Record<string, string>>({});
\end{lstlisting}


Хэрэглэгчийн мэдээллийг татах
\begin{lstlisting}[language=Typescript, caption=Хэрэглэгчийн мэдээлэл татах useEffect, frame=single]
	useEffect(() => {
    const fetchUserData = async () => {
      if (!id || !session?.user?.token) return;

      try {
        setLoading(true);
        const response = await req.GET(
          `/admin/users?id=${id}`,
          session.user.token
        );
        setFormData({
          ...response,
          updated_by: session.user.id || "",
        });
      } catch (error) {
        toast({
          title: "Алдаа",
          description: "Хэрэглэгчийн мэдээлэл авахад алдаа гарлаа",
          variant: "destructive",
        });
      } finally {
        setLoading(false);
      }
    };

    fetchUserData();
  }, [id, session?.user?.id, session?.user?.token]);
\end{lstlisting}

Оролтын өөрчлөлт хянах handleChange функц
\begin{lstlisting}[language=Typescript, caption=handleChange, frame=single]
	const handleChange = (e: React.ChangeEvent<HTMLInputElement>) => {
    const { name, value, type, checked } = e.target;
    setFormData((prev) => ({
      ...prev,
      [name]: type === "checkbox" ? checked : value,
    }));
  };
\end{lstlisting}

Хүсэлт илгээх handleSubmit функц
\begin{lstlisting}[language=Typescript, caption=handleSubmit, frame=single]
	const handleSubmit = async (e: React.FormEvent) => {
    e.preventDefault();

    try {
      setLoading(true);
      const response = await req.PUT(
        `/admin/users/${id || ""}`,
        session?.user?.token || "",
        {
          ...formData,
          active_start_date: formData.active_start_date || null,
          active_end_date: formData.active_end_date || null,
        }
      );

      if (response) {
        toast({
          title: "Амжилттай",
          description: "Мэдээлэл шинэчлэгдлээ",
        });
      }
    } catch (error) {
      console.error("Error submitting form:", error);
      toast({
        title: "Алдаа",
        description: "Гэнэтийн алдаа гарлаа",
        variant: "destructive",
      });
    } finally {
      setLoading(false);
    }
  };
\end{lstlisting}

Формын UI бүтэц
\begin{lstlisting}[language=Typescript, caption=UI бүтэц, frame=single]
	return (
    <div className="py-8 px-4">
      <Card>
        <CardHeader>
          <CardTitle className="text-2xl">
            {id ? "Хэрэглэгч засах" : "Шинэ хэрэглэгч нэмэх"}
          </CardTitle>
        </CardHeader>
        <CardContent>
          <form onSubmit={handleSubmit} className="space-y-4">
            <div className="space-y-6">
              <h3 className="text-lg font-medium">Хувийн мэдээлэл</h3>
              <div className="grid grid-cols-1 md:grid-cols-2 gap-4">
                <div>
                  <Label htmlFor="first_name">Нэр</Label>
                  <Input
                    id="first_name"
                    name="first_name"
                    type="text"
                    value={formData.first_name}
                    onChange={handleChange}
                    className={errors.first_name ? "border-red-500" : ""}
                  />
                  {errors.first_name && (
                    <p className="text-red-500 text-sm mt-1">
                      {errors.first_name}
                    </p>
                  )}
                </div>
                <div>
                  <Label htmlFor="last_name">Овог</Label>
                  <Input
                    id="last_name"
                    name="last_name"
                    type="text"
                    value={formData.last_name}
                    onChange={handleChange}
                    className={errors.last_name ? "border-red-500" : ""}
                  />
                  {errors.last_name && (
                    <p className="text-red-500 text-sm mt-1">
                      {errors.last_name}
                    </p>
                  )}
                </div>
                <div>
                  <Label htmlFor="login_id">Нэвтрэх ID</Label>
                  <Input
                    id="login_id"
                    name="login_id"
                    type="text"
                    value={formData.login_id}
                    onChange={handleChange}
                    className={errors.login_id ? "border-red-500" : ""}
                  />
                  {errors.login_id && (
                    <p className="text-red-500 text-sm mt-1">
                      {errors.login_id}
                    </p>
                  )}
                </div>
                {!id && (
                  <div>
                    <Label htmlFor="password">Нууц үг</Label>
                    <Input
                      id="password"
                      name="password"
                      type="password"
                      value={formData.password}
                      onChange={handleChange}
                      className={errors.password ? "border-red-500" : ""}
                    />
                    {errors.password && (
                      <p className="text-red-500 text-sm mt-1">
                        {errors.password}
                      </p>
                    )}
                  </div>
                )}
              </div>
            </div>
            <div className="space-y-2">
              <h3 className="text-lg font-medium">Холбоо барих</h3>
              <div className="grid grid-cols-1 md:grid-cols-2 gap-4">
                <div>
                  <Label htmlFor="email_work">Ажлын имэйл</Label>
                  <Input
                    id="email_work"
                    name="email_work"
                    type="email"
                    value={formData.email_work}
                    onChange={handleChange}
                    className={errors.email_work ? "border-red-500" : ""}
                  />
                  {errors.email_work && (
                    <p className="text-red-500 text-sm mt-1">
                      {errors.email_work}
                    </p>
                  )}
                </div>
                <div>
                  <Label htmlFor="email_personal">Хувийн имэйл</Label>
                  <Input
                    id="email_personal"
                    name="email_personal"
                    type="email"
                    value={formData.email_personal}
                    onChange={handleChange}
                    className={errors.email_personal ? "border-red-500" : ""}
                  />
                  {errors.email_personal && (
                    <p className="text-red-500 text-sm mt-1">
                      {errors.email_personal}
                    </p>
                  )}
                </div>
                <div>
                  <Label htmlFor="phone_number_work">Ажлын дугаар</Label>
                  <Input
                    id="phone_number_work"
                    name="phone_number_work"
                    type="tel"
                    value={formData.phone_number_work}
                    onChange={handleChange}
                  />
                </div>
                <div>
                  <Label htmlFor="phone_number_personal">Хувийн дугаар</Label>
                  <Input
                    id="phone_number_personal"
                    name="phone_number_personal"
                    type="tel"
                    value={formData.phone_number_personal}
                    onChange={handleChange}
                  />
                </div>
              </div>
            </div>
            <div className="space-y-2">
              <h3 className="text-lg font-medium">Төлөв</h3>
              <div className="grid grid-cols-1 md:grid-cols-2 gap-4 items-center">
                <div className="flex items-center space-x-2">
                  <input
                    id="is_active"
                    name="is_active"
                    type="checkbox"
                    checked={formData.is_active}
                    onChange={handleChange}
                    className="h-4 w-4 border-gray-300 rounded"
                  />
                  <Label htmlFor="is_active">Идэвхтэй эсэх</Label>
                </div>
                <div className="space-x-2">
                  <Label>Ажиллах хугацаа:</Label>
                  <DatePickerWithRange
                    value={{
                      from: formData.active_start_date ? new Date(formData.active_start_date) : undefined,
                      to: formData.active_end_date ? new Date(formData.active_end_date) : undefined,
                    }}
                    onChange={(date) => {
                      setFormData((prev) => ({
                        ...prev,
                        active_start_date: date?.from ? date.from.toISOString(): "",
                        active_end_date: date?.to ? date.to.toISOString() : "",
                      }));
                    }}
                  />
                </div>
              </div>
            </div>
            <UserRole userId={id!} />
            <div>
              <Button type="submit" className="w-full" disabled={loading}>
                {loading ? "Уншиж байна..." : id ? "Засах" : "Нэмэх"}
              </Button>
            </div>
          </form>
        </CardContent>
      </Card>
    </div>
  );
\end{lstlisting}
\pagebreak