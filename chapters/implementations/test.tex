\section{Golang Test хэрэгжүүлэлт}

Энэхүү хэсэгт бид Go хэл дээрх үйлдлүүдийг тест хийх зориулалттай функцуудыг хэрэгжүүлсэн болно. Тестүүд нь хэрэглэгчийн мэдээллийг нэмэх, авах, шинэчлэх, устгах зэрэг CRUD үйлдлүүдийг шалгана.



\subsection{Тестийн орчин}

Тестийн орчин байгуулахдаа өгөгдлийн санг анхны нөхцөлд нь оруулж, дараа нь алдаагүйгээр бүх функцуудыг туршина. Энэ зорилгоор дараах функцүүдийг ашиглав:

\begin{lstlisting}[language=Go, caption=Test Database Setup, frame=single]
func setupTestDB() *gorm.DB {
    test.ConnectTestDatabase()
    config.DB.Exec("TRUNCATE TABLE users RESTART IDENTITY CASCADE") 
    return config.DB
}
\end{lstlisting}

Энэхүү функц нь тестийн өгөгдлийн санг тохируулж, хэрэглэгчийн хүснэгтийг шинэчилнэ.

\subsection{Тестийн функцууд}

\begin{itemize}
    \item \texttt{TestCreateUser}: Шинэ хэрэглэгч үүсгэх үйлдлийг шалгана.
    \item \texttt{TestGetAllUsers}: Бүх хэрэглэгчийн мэдээллийг авах функцыг туршина.
    \item \texttt{TestUpdateUser}: Хэрэглэгчийн мэдээллийг шинэчлэх үйлдлийг шалгана.
    \item \texttt{TestDeleteUser}: Хэрэглэгчийн мэдээллийг устгах үйлдлийг туршина.
    \item \texttt{TestGetUserByID}: Хэрэглэгчийн ID-ээр мэдээлэл авах үйлдлийг шалгана.
\end{itemize}

\subsubsection{TestCreateUser}

Энэхүү тест нь шинэ хэрэглэгчийг үүсгэж, үүсгэх үед гарах алдааг шалгах бөгөөд хэрэглэгчийн нууц үгийг хашлах явцыг шалгана.

\begin{lstlisting}[language=Go, caption=Test Create User Function, frame=single]
func TestCreateUser(t *testing.T) {
    db := setupTestDB()
    defer test.CloseTestDatabase()

    loginID := "johndoe"
    dropExistingUser(db, loginID)

    r := httptest.NewRequest(http.MethodPost, "/", nil)
    user := createSampleUser(loginID)

    createdUser, err := services.CreateUser(user, r)

    assert.NoError(t, err, "Error creating user")
    assert.NotZero(t, createdUser.ID, "User ID should not be zero")
    assert.NotEqual(t, "password123", createdUser.Password, "Password should be hashed")

    var fetchedUser models.User
    err = db.First(&fetchedUser, "id = ?", createdUser.ID).Error
    assert.NoError(t, err, "Error fetching user")

    assert.Equal(t, "John", fetchedUser.FirstName, "First name mismatch")
    assert.Equal(t, "Doe", fetchedUser.LastName, "Last name mismatch")
    assert.NotEqual(t, "password123", fetchedUser.Password, "Password should be hashed")
}
\end{lstlisting}

\subsubsection{TestGetAllUsers}

Энэхүү тест нь өгөгдлийн сан дахь бүх хэрэглэгчийн мэдээллийг авах үйлдлийг шалгана. Мөн хэрэглэгчдийн тоог болон мэдээллийг баталгаажуулна.

\begin{lstlisting}[language=Go, caption=Test Get All Users Function, frame=single]
func TestGetAllUsers(t *testing.T) {
    db := setupTestDB()
    defer test.CloseTestDatabase()

    db.Exec("TRUNCATE TABLE users RESTART IDENTITY CASCADE")

    users := []models.User{
        {FirstName: "John", LastName: "Doe"},
        {FirstName: "Jane", LastName: "Smith"},
    }
    for _, user := range users {
        db.Create(&user)
    }

    r := httptest.NewRequest(http.MethodGet, "/users", nil)
    allUsers, err := services.GetAllUsers(r)

    assert.NoError(t, err, "Error fetching all users")
    assert.Len(t, allUsers, len(users), "Unexpected number of users fetched")

    for i, fetchedUser := range allUsers {
        assert.True(t, i < len(users), "Index %d out of range for expected users", i)
        assert.Equal(t, users[i].FirstName, fetchedUser.FirstName, 
            "First name mismatch. Expected: %s, Got: %s", users[i].FirstName, fetchedUser.FirstName)
        assert.Equal(t, users[i].LastName, fetchedUser.LastName, 
            "Last name mismatch. Expected: %s, Got: %s", users[i].LastName, fetchedUser.LastName)
    }
}
\end{lstlisting}

\subsubsection{TestUpdateUser}

Энэхүү тест нь хэрэглэгчийн мэдээллийг шинэчлэх үйлдлийг шалгадаг.

\begin{lstlisting}[language=Go, caption=Test Update User Function, frame=single]
func TestUpdateUser(t *testing.T) {
    db := setupTestDB()
    defer test.CloseTestDatabase()

    // Create sample user
    originalUser := createSampleUser("johndoe")
    db.Create(&originalUser)

    // Prepare updated user data
    updatedUserData := models.User{
        FirstName: "Jane",
        LastName:  "Smith",
        Password:  "newpassword123",
    }

    r := httptest.NewRequest(http.MethodPut, "/user", nil)
    updatedUser, err := services.UpdateUser(originalUser.ID, updatedUserData, r)

    assert.NoError(t, err, "Error updating user")
    assert.Equal(t, updatedUserData.FirstName, updatedUser.FirstName, "First name mismatch")
    assert.Equal(t, updatedUserData.LastName, updatedUser.LastName, "Last name mismatch")
    assert.NotEqual(t, "newpassword123", updatedUser.Password, "Password should be hashed")

    var fetchedUser models.User
    err = db.First(&fetchedUser, "id = ?", originalUser.ID).Error
    assert.NoError(t, err, "Error fetching updated user")

    assert.Equal(t, updatedUserData.FirstName, fetchedUser.FirstName, "First name mismatch in database")
    assert.Equal(t, updatedUserData.LastName, fetchedUser.LastName, "Last name mismatch in database")
    assert.NotEqual(t, "newpassword123", fetchedUser.Password, "Password should be hashed in database")
}
\end{lstlisting}
\pagebreak
