\section{GORM хэрэгжүүлэлт}

Уг функцуудыг шаардлагын дагуу алхам алхмаар хэрэгжүүлсэнээр GORM-ийг ашиглах анхан шатны туршлагатай болох зорилготой байсан ба цаашид ажилласан төсөл дээр GORM-ийн дотоод логик, өгөгдлийн сангийн холболт, тайлбар хийх, өгөгдөл хадгалах болон шинэчлэх процессуудын ажиллагааг ойлгох, тодорхой хэмжээнд практик мэдлэгийг цуглуулж чадсан.

Сурах ур чадвар: 
\begin{itemize} 
	\item GORM-ийг ашиглах чадвараа нэмэгдүүлэх 
	\item Өгөгдлийн сантай харьцах үйлдлүүд (GET, POST, UPDATE, DELETE) болон тэдгээрийн процессийг ойлгох 
	\item Хэрэглэгчийн логийг бүртгэх болон алдааг барих чадвар эзэмших 
	\item CRUD үйлдлүүдийг GORM ашиглан бичих 
	\item Горм-ийг ашиглахдаа data validation болон password hashing-ийн талаар мэдлэгтэй болох \end{itemize}

Функцийн шаардлага: 
\begin{itemize} 
	\item GORM ашиглан хэрэглэгчийн мэдээллийг авах, нэмэх, засах, устгах 
	\item Хэрэглэгчийн нэвтрэх ID-г шалгах функц бичих 
	\item Алдааг бүртгэх болон лог хийх функц ашиглах 
	\item GORM-ийн дотоод функцууд болон query үйлдлийг ойлгох 
	\item Хэрэглэгчийн мэдээлэл устгах үед хамааралтай мэдээллийг устгах
	\item Controller  давхаргыг зөв зохион байгуулах
\end{itemize}
	
\subsection{User Management Functions}

Энэхүү хэсэгт хэрэглэгчийн мэдээллийг удирдах хэд хэдэн функцийг оруулсан болно. Үүнд хэрэглэгчийн мэдээллийг авах, шинэчлэх, устгах, болон логины ID-г шалгах үйлдлүүд орно.

\begin{lstlisting}[language=Go, caption=Get all users function, frame=single]
func GetAllUsers(r *http.Request) ([]models.User, error) {
	var users []models.User
	if err := config.DB.Find(&users).Error; err != nil {
		log.Error("Failed to fetch users", map[string]interface{}{"error": err.Error()}, r)
		return nil, err
	}
	return users, nil
}
\end{lstlisting}

Энэ функц нь бүх хэрэглэгчдийн мэдээллийг өгөгдлийн сангаас авч, ямар нэгэн алдаа гарсан тохиолдолд тухайн алдааг логлош болно.

\begin{lstlisting}[language=Go, caption=Create user function, frame=single]
func CreateUser(user models.User, r *http.Request) (models.User, error) {
	existingUser := config.DB.Where("login_id = ?", user.LoginID).First(&models.User{})
	if existingUser.RowsAffected > 0 {
		return models.User{}, errors.New("user already exists")
	}

	hashedPassword, err := utils.HashPassword(user.Password)
	if err != nil {
		return models.User{}, err
	}

	user.Password = hashedPassword

	if err := config.DB.Create(&user).Error; err != nil {
		return models.User{}, err
	}
	return user, nil
}
\end{lstlisting}

Шинэ хэрэглэгчийг үүсгэх үед, хэрэглэгчийн ID өмнө нь байгаа эсэхийг шалгаад, байсан бол алдаа буцаана. Мөн хэрэглэгчийн нууц үгийг хуулаад өгөгдлийн сан дээр хадгална.

\begin{lstlisting}[language=Go, caption=Update user function, frame=single]
func UpdateUser(id string, user models.User, r *http.Request) (models.User, error) {
	hashedPassword, err := utils.HashPassword(user.Password)
	if err != nil {
		return models.User{}, err
	}

	user.Password = hashedPassword
	if err := config.DB.Model(user).Where("id = ?", id).Updates(user).Error; err != nil {
		return models.User{}, err
	}
	return user, nil
}
\end{lstlisting}

Энэ функц нь хэрэглэгчийн мэдээллийг шинэчилж, нууц үгийг шинэчлэх шаардлагатай бол шинэчилнэ.

\begin{lstlisting}[language=Go, caption=Delete user function, frame=single]
func DeleteUser(id string, r *http.Request) error {
	if err := config.DB.Where("user_id = ?", id).Delete(&models.UserRole{}).Error; err != nil {
		return err
	}

	if err := config.DB.Where("id = ?", id).Delete(&models.User{}).Error; err != nil {
		return err
	}

	return nil
}
\end{lstlisting}

Хэрэглэгчийг устгах үед түүний хэрэглэгчийн үүрэг болон хэрэглэгчийн мэдээллийг устгана.

\begin{lstlisting}[language=Go, caption=Get user by ID function, frame=single]
func GetUserByID(id string, r *http.Request) (*models.User, error) {
	log.Info("Fetching user by ID from the database", map[string]interface{}{"id": id}, r)

	var user models.User

	err := config.DB.Where("id = ?", id).First(&user).Error
	if err != nil {
		if errors.Is(err, gorm.ErrRecordNotFound) {
			log.Warn("User not found", map[string]interface{}{"id": id}, r)
			return nil, nil
		}
		log.Error("Database query error", map[string]interface{}{"error": err.Error()}, r)
		return nil, err
	}

	log.Info("User fetched successfully", map[string]interface{}{"id": id}, r)
	return &user, nil
}
\end{lstlisting}

Энэхүү функц нь хэрэглэгчийн ID-г ашиглан мэдээллийг өгөгдлийн сангаас авдаг. Хэрэв хэрэглэгч олдвол түүний мэдээллийг буцаана.

\begin{lstlisting}[language=Go, caption=Check if login ID exists function, frame=single]
func CheckLoginIDExists(loginID string, r *http.Request) (bool, error) {
	var user models.User
	result := config.DB.Where("login_id = ?", loginID).First(&user)

	if result.Error != nil {
		if errors.Is(result.Error, gorm.ErrRecordNotFound) {
			return false, nil
		}
		log.Error("Failed to check login_id", map[string]interface{}{"error": result.Error.Error()}, r)
		return false, result.Error 
	}

	return true, nil 
}
\end{lstlisting}

Энэ функц нь өгөгдлийн сан дээрх login ID-гийн байршлыг шалгадаг. Хэрэв байвал true, байхгүй бол false буцаана.
\pagebreak
