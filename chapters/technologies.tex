\section{Gitlab}
GitLab нь Git дээр суурилсан DevOps платформ бөгөөд программ хангамж хөгжүүлэх, тест хийх, нэвтрүүлэх процессыг автоматжуулан нэг дор удирдах боломжийг олгодог. GitLab нь CI/CD (Continuous Integration/Continuous Deployment) багтаасан байдаг ба энэ нь хөгжүүлэгчид өөрсдийн кодыг турших, хянах, шинэчлэлтүүдийг хурдан бөгөөд найдвартай нэвтрүүлэхэд тусалдаг. Хөгжүүлэгчид өөрсдийн төслийг GitLab дээр байршуулж, хамтын ажиллагааг хөнгөвчилж, кодын чанарыг сайжруулах боломжтой байдаг.

\section{NextJS}
Next.js нь React-д суурилсан, сервер талын рендэрлэг SSR (Server-side rendering), статик сайт үүсгэгч (Static Site Generation), API маршрут боловсруулагч зэрэг олон төрлийн функцуудыг агуулсан хүчирхэг framework юм. Next.js нь хурдан, SEO (search engine optimization)-д ээлтэй веб аппликейшнуудыг бүтээхэд туслах зорилготой. Энэ нь хуудасны динамик маршрут, API endpoint-ууд, автомат статик зэрэг олон төрлийн функцуудыг агуулдаг. Next.js нь хөгжүүлэгчдэд илүү хялбар бөгөөд хурдан хөгжүүлэлтийг хангахын зэрэгцээ, веб аппликейшны гүйцэтгэлийг сайжруулдаг. Мөн Next.js нь TypeScript-ийг дэмжиж, хөгжүүлэгчдэд илүү найдвартай, уншигдахуйц код бичих боломжийг олгодог.

\section{Golang}
Golang буюу Go нь Google-ээс гаргасан, хурдан, найдвартай, энгийн синтакс бүхий программчлалын хэл юм. Go нь системийн программчлал, сүлжээний программчлал, cloud computing, микросервис архитектур зэрэг олон төрлийн хэрэглээнд тохиромжтой. Go хэл нь статик төрөлтэй, garbage collection-тэй, multi-threading-ийг хялбаршуулсан goroutine-уудтай байдаг. Энэ нь хөгжүүлэгчдэд өндөр гүйцэтгэлтэй программуудыг бүтээх боломжийг олгодог. Go нь мөн стандарт сангуудын багцтай байдаг бөгөөд энэ нь хөгжүүлэгчдэд нэмэлт сангүйгээр олон төрлийн ажиллагааг хэрэгжүүлэх боломжийг олгодог.

Go хэл нь мөн GORM гэдэг объектын холбоосын (ORM) санг дэмждэг бөгөөд
энэ нь Go хэл дээрх өгөгдлийн сантай ажиллах процессыг
хялбаршуулдаг. GORM нь хөгжүүлэгчдэд SQL болон өгөгдлийн сан
холбохыг хялбаршуулж, объектуудыг шууд өгөгдлийн сангийн хүснэгтүүдэд
хамааруулан ажиллах боломжийг олгодог. Энэ нь CRUD (Create, Read
Update, Delete) үйлдлүүдийг гүйцэтгэх, өгөгдлийн миграци, холболт
хийх зэрэг үүргийг гүйцэтгэх боломжийг бүрдүүлдэг. GORM-ийн ашиглах
нь Go-ийн синтаксийн энгийн байдал, хурдтай холбогдсон, мөн өгөгдлийн
сангийн бүхий л үйлдлүүдийг энгийн, хурдан гүйцэтгэх боломжоор хангадаг.



\section{PostgreSQL}
PostgreSQL нь хүчирхэг, нээлттэй open source объект хандлагат database бөгөөд сүүлийн 30 гаруй жилийн турш идэвхтэй хөгжиж байгаа найдвартай ажиллагаатай мэдээллийн сангийн программ юм. 1986 онд Калифорни их сургуулийн “Postgres” төслийн нэг хэсэг болж хөгжсөн түүхтэй. PostgreSQL ихэнх үйлдлийн системүүд дээр ажилладаг ба PostGIS гэсэн орон зайн мэдээллийн сангийн нэмэлт хэрэгсэлтэйгээрээ бусад мэдээллийн сангийн программуудаас давуу талтай. Мэдээллийн сан бол таны мэдээллийг найдвартай хадгалах үүрэгтэй. PostgreSQL мэдээллийн сан нь SQL хэлний дийлэнх стандартууд болон орчин үед хэрэглэгдэж байгаа ихэнх мэдээллийн сангийн цогц асуулгууд буюу queries, гадаад түлхүүр буюу foreign keys, триггер командууд, өөрчлөлт хийх боломжтой view, мэдээ дамжуулах хурд, гэх мэт боломжуудыг өөртөө агуулсан байдаг. Мөн хэрэглэгч нь мэдээний төрөл, функцууд, үйлдлүүд, индексийн арга зэргийг өөрийн хэрэгцээнд зориулан хөгжүүлэх боломжтой нээлттэй эх үүсвэрийн мэдээллийн сан юм. PostgreSQL системийн үндсэн архитектурыг харвал клиент / серверийн загварыг ашигладаг.

\section{Postman}
Postman нь API хөгжүүлэлт, тест хийх, баримтжуулалт хийхэд ашиглагддаг хэрэгсэл юм. Энэ нь хөгжүүлэгчдэд API-уудыг хялбархан турших, өөрчлөлт оруулах, багийн гишүүдтэй хамтран ажиллах боломжийг олгодог. Postman нь REST, GraphQL, SOAP зэрэг олон төрлийн API-уудыг дэмждэг бөгөөд хөгжүүлэгчдэд API-ийн хүсэлт, хариуг хялбархан үүсгэх, илгээх, хариу үйлдлийг шалгах боломжийг олгодог. Мөн Postman нь автомат тест үүсгэх, тест сценари бичих, API-ийн гүйцэтгэлийг шалгах боломжийг олгодог. Энэ нь хөгжүүлэлтийн явцыг хурдасгах, API-ийн найдвартай байдлыг нэмэгдүүлэхэд тусалдаг.

\section{Docker}
Docker нь аппликейшнуудыг контейнер хэлбэрээр багцлах, ажиллуулах, түгээхэд ашиглагддаг платформ юм. Docker нь аппликейшны хөгжүүлэлт, тест хийх, production орчинд хүргэх явцыг хялбаршуулдаг. Контейнерчилал нь аппликейшны бүх шаардлагатай нөөцийг (код, санууд, системийн хэрэгслүүд) нэг дор багцлах боломжийг олгодог бөгөөд энэ нь аппликейшныг ямар ч орчинд ажиллуулах боломжийг олгодог. Docker нь хөгжүүлэгчдэд аппликейшны орчны тогтвортой байдлыг хангах, аппликейшныг хурдан, найдвартайгаар хүргэх боломжийг олгодог. Мөн Docker нь микросервис архитектурт тохиромжтой бөгөөд энэ нь орчин үеийн cloud-native аппликейшнуудыг хөгжүүлэхэд тусалдаг.